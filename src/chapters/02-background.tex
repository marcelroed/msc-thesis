\documentclass[../main.tex]{subfiles}

% Document
\begin{document}
    \chapter{Background}\label{ch:background}

    \section{Diffusion and Partial Differential Equations}\label{sec:diffusion}
    A partial differential equation (PDE) is a set of constraints on the partial derivatives of an underlying function, often parameterized in terms of a different function.
    PDEs often take the form of expressions using their $n$th order partial derivatives, with some examples following in the table below.
    \todo{table of a few well known PDEs}
    The process of solving a PDE is ``what are the functions that satisfy these constraints?''
    The heat equation, also known as the diffusion equation, is a classic example of a PDE that shows up in many domains, and will be used in the  we will study in this thesis.
    Developed by Joseph Fourier in the early 18th century~\cite{narasimhanFourierHeatConduction1999}, the purpose of the differential equation was to study the transfer of heat, but it has found uses in a broad spectrum of applications, e.g.\ studying particle dynamics, quantum mechanics and economics.
    Importantly,

    A somewhat general yet intuitive description of diffusion would be the flow of ``substance'' from high levels to lower levels at a rate that is proportional to the gradient of the level of substance at a given area (for the heat equation, substitute substance for temperature, and flow of substance for heat).
    The basic equation that expresses this is somewhat non-obvious to those who haven't studied it, written as

    \begin{equation}
        \parpar ut = \alpha\Delta u,\label{eq:basic-heat}
    \end{equation}

    for our ``substance'' function $u(x, t)$, (thermal) diffusivity constant $\alpha$, where $\Delta = \nabla^2$ is the Laplacian.
    To understand why the expression uses the Laplacian, an operator recovering the curvature of a function, consider the fact that a point at position $x$ at time $t + \epsilon$ for sufficiently small $\epsilon > 0$ will be moved if its
    neighboring points are higher than itself is.
    If the gradient is positive in one direction, the flow of heat could be cancelled by a negative gradient in a different direction.
    We need to measure the differences in gradients around the point $x$ at time $t$ to find out the ``substance'' change per unit time at instant $t$.

    The description in words above is formulated in \autoref{prop:diffusion-as-heat}:

    \begin{proposition}\label{prop:diffusion-as-heat}
        In the diffusion equation, the function value $u(x, t)$ at position $x$ will change over time $t$ toward the values of its surroundings.
    \end{proposition}

    \autoref{eq:basic-heat} belies another intuition for what the heat equation means:

    \begin{proposition}\label{prop:diffusion-as-curvature}
        In the diffusion equation, function values change to minimize their spatial curvature.
    \end{proposition}

    \autoref{prop:diffusion-as-heat} follows closely the standard interpretation of the heat equation.
    This can be thought of heat or particles flowing from positions where there is more to fill positions which have less.
    In the absence of sources and drains of heat or particles (referred to as homogeneous systems), we have

    Instead, \autoref{prop:diffusion-as-curvature} can be thought of as minimization of curvature.
    In this interpretation we are moving each point on the function with a rate proportional to its curvature.
    A property of the solutions to this differential equation is much more obvious here than in the other interpretation.
    Namely: when the curvature is zero, ergo the function is locally linear, the function value will stay constant.

%    In all generality, the diffusion equation may be written as
%    \begin{equation}\label{eq:diffusion-general}
%    \parpar{\phi(\bm r, t)}t
%    \end{equation}
%
%    For many use-cases, the
%
%    % Insert equation
%    % ,
%
%    and this simplification can be interpreted in two key ways.

    \subsection{Numerical Solutions to Partial Differential Equations}\label{subsec:solving-partial-differential-equations}

    There are several ways of solving PDEs, but central to this thesis is Runge-Kutta (RK) methods, the simplest of which being Euler's method.

    Given an initial value problem of the form
    \begin{equation}
        \dd yt = f(t, y), \quad y(t_0) = y_0,\label{eq:ivp}
    \end{equation}
    we

    \subsubsection{Simple Solvers}\label{subsubsec:simple-solvers}
    \todo{Write up how simple PDEs can be solved}

    \subsubsection{Stiff Equations}\label{subsubsec:stiff-equations}
    The methods described above work well for simple functions, but may fail spectacularly when solving PDEs that are expressed in complex functions $f$, for instance when parameterized by a neural network, a case we will come back to in \autoref{subsec:neural-differential-equations} as well as in \autoref{sec:graph-neural-diffusion} and \autoref{sec:the-gradient-flow-framework}.
    

    \subsubsection{Adaptive Methods and Tricks for Stiff Equations}\label{subsubsec:adaptive-methods}
    The implementation (see \todo{link section on solving the solver}), the inner workings of these methods is beyond the scope of this thesis.

    In particular use of higher order adaptive~\cite{pressAdaptiveStepsizeRungeKutta1992a} RK methods like Tsitouras 5(4)~\cite{tsitourasRungeKuttaPairs2011} and Dormand Prince 5(4)~\cite{dormandFamilyEmbeddedRungeKutta1980}

    \subsection{Neural Differential Equations}\label{subsec:neural-differential-equations}
    In his thesis on the matter~\cite{kidgerNeuralDifferentialEquations2022}, Patrick Kidger concisely defines a neural differential equation as ``a differential equation using a neural network to parameterize the vector field''.
    The canonical example of this is the standard neural ODE
    \begin{equation}
        y(0) = y_0 \quad \frac{\mathrm dy}{\mathrm dt}(t) = f_{\theta}(t, y(t)).\label{eq:neural-ode}
    \end{equation}
    In this equation, $\theta$ represents the weights of the flow field neural network $f_{\theta}: \R \times \R^{d_1 \times \cdots \times d_k} \to \R^{d_1\times \cdots \times d_k}$.



    \section{Graph Neural Networks}\label{sec:graph-neural-networks}
    Graph Neural Network (GNN), is a collective term for neural network architectures that run on graphs as their input data.

    \subsection{Graph Learning Tasks}\label{subsec:graph-learning-tasks}
    Using graph input, one may consider several prediction tasks.
    Often, these are separated into which aspects of the graph they are trying to make predictions about.
    We have
    \begin{itemize}
        \item Node-level tasks -- Predict some features per node on the input graph.
        Often this would be classification of the node features, for instance classifying fake accounts in a social media graph.
        If we reorder the nodes (the order in which they are organized in the input representation), we expect reordered outputs.
        \item Edge-level tasks -- Predict features on existing or potential edges on the input graph.
        This might involve which edges might be missing, or the strength of a relationship.
        Again, reordering the input graph should appropriately reorder the output of these predictions.
        \item Graph-level tasks -- Predict features on the entire graph.
        If our graph is a molecule, we could for instance try to predict material properties.
        Reordering the nodes and edges should in this instance have no effect on our prediction.
    \end{itemize}

    \section{Message Passing Neural Networks}\label{sec:message-passing-neural-networks}
    Message passing neural networks~\cite{gilmerNeuralMessagePassing2017} (MPNNs) are a subclass of graph neural networks that conform to the following equations
    \begin{align}
        m^{t+1}_v &= \sum_{w \in N(v)} M_t(h^t_v, h^t_w, e_{vw})\label{eq:mpnn-message}\\
        h^{t+1}_v &= U_t(h_v^t, m_v^{t + 1}),\label{eq:mpnn-update}\\
    \end{align}
    where $m^{t+1}_v$ is the message computed for timestep $t + 1$ at node $v$, $N(v)$ is the set of neighbors of $v$, $h^t_x$ is the state of node $x$ at timestep $t$ and $e_{vw}$ is the edge weight from node $v$ to node $w$.
    $M_t$ and $U_t$ are the message function and the node update function at time $t$ respectively, and this time variable is usually used as an index for the current layer of a feed-forward neural network.

    One could also consider global tasks on

    \subsection{Graph Data and Datasets}\label{subsec:graph-datasets}
    Unlike images or time series data, connections and thus (first order) interactions between features in an input graph are not fixed between graphs, nor are they arbitrary.



\end{document}
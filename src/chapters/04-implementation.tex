%! Author = marcel
%! Date = 11/09/2022

% Preamble
\documentclass[../main.tex]{subfiles}


% Document
\begin{document}
    \chapter{Implementation}\label{ch:implementation}
    When implementing the



    \section{}\label{sec:technical-hurdles}

    \subsection{Choosing a framework}\label{subsec:choosing-a-framework}
    There are several possible choices for a method for

    \subsection{Jax}\label{subsec:jax}
    Using Jax presents several advantages for the implementation of the neural architecture.
    While in some aspects the tooling is immature, the extensibility and raw speed of Jax has allowed for well optimized methods that are not easily available even in the more mature PyTorch.

    \subsection{Fuzzy Clustering}\label{subsec:implementing-fuzzy-clustering}
    The fuzzy clustering algorithm defined in \autoref{subsec:fuzzy-clustering} is
    Clustering on highly parallel hardware like a GPU or TPU


    \section{Datasets}\label{sec:datasets}
    \subsection{Standard Datasets}\label{subsec:standard-datasets}
    In order to get some idea of how this method performs on real datasets

    \subsection{Synthetic Datasets}\label{subsec:synthetic-datasets}
    In addition, we construct some ideal and some adverserial datasets that show best and worst-case situations.
    In \autoref{ch:discussion} there's some in-depth discussion on results on these datasets and what they mean for the model as a whole.

    The HomophilicAberration dataset \ldots \todo{fill in details on how each dataset is constructed and motivate the construction}

    \todo{Maybe move this to method?}


\end{document}